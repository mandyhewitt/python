
\begin{DoxyRefList}
\item[\label{bug__bug000011}%
\Hypertarget{bug__bug000011}%
Global \hyperlink{templates_8h_acbbafed45b10980bb86fe7950daa0604}{adiabatic\+\_\+cooling} (Wind\+Ptr one, double t)]The statements here about not calling this routine are odd. There is no reason that the routine cannot be called any time since it does not populate cool\+\_\+adiabatic. And what really shold happen if we do not want to use cool\+\_\+adiabatic is that we should check when we want to calculate the total cooling that it is not done. 
\item[\label{bug__bug000003}%
\Hypertarget{bug__bug000003}%
Global \hyperlink{templates_8h_a3513a64162c15d3d3d101c3d5ea2367f}{bands\+\_\+init} (int imode, struct xbands $\ast$band)]At some point some of the banding needs to be tested systematically. We have tended to find a setup that we like and just use it. 
\item[\label{bug__bug000045}%
\Hypertarget{bug__bug000045}%
Global \hyperlink{templates_8h_afb47deaf80a8f3a6f07ab91d843d960e}{bl\+\_\+init} (double lum\+\_\+bl, double t\+\_\+bl, double freqmin, double freqmax, int ioniz\+\_\+or\+\_\+final, double $\ast$f)]At present bl\+\_\+init assumes a BB regardless of the spectrum. This is not really correct, and is different for what is done in initializing the star 
\item[\label{bug__bug000007}%
\Hypertarget{bug__bug000007}%
Global \hyperlink{templates_8h_a30c0717640ef307cbe5489ec9ef911f5}{calc\+\_\+cdf\+\_\+gradient} (Cdf\+Ptr cdf)]X\+XX N\+SH -\/ the way that the ends are dealt with is not great -\/ we should really try to come up with an extrapolation rather than just fill in the same gradients for the second and penultimate cells into the first and last cells. 
\item[\label{bug__bug000006}%
\Hypertarget{bug__bug000006}%
File \hyperlink{cdf_8c}{cdf.c} ]For reasons, which are currently unclear there are differences in the number of points maintained in the cdfs for different generation methods. 
\item[\label{bug__bug000032}%
\Hypertarget{bug__bug000032}%
Global \hyperlink{templates_8h_a834b9001d849a521f4dbf7a0dcb0d012}{check\+\_\+convergence} (void)]It would make sense to write this information to a separate file so that plots of the rate of convergence could be easily made. 
\item[\label{bug__bug000032}%
\Hypertarget{bug__bug000032}%
Global \hyperlink{templates_8h_a834b9001d849a521f4dbf7a0dcb0d012}{check\+\_\+convergence} (void)]It would make sense to write this information to a separate file so that plots of the rate of convergence could be easily made. 
\item[\label{bug__bug000004}%
\Hypertarget{bug__bug000004}%
Global \hyperlink{templates_8h_a11871d7df7f3167db92bef8c8ef98926}{check\+\_\+fmax} (double fmin, double fmax, double temp)]This routine calls several variables that it does not use. It is also unclear that it is needed. 
\item[\label{bug__bug000030}%
\Hypertarget{bug__bug000030}%
Global \hyperlink{templates_8h_afa8b5c03286c90c5365b874967e81b33}{check\+\_\+xsections} (void)]A careful look at this routine is warranted as get\+\_\+atomic\+\_\+data has changed over the years. 
\item[\label{bug__bug000030}%
\Hypertarget{bug__bug000030}%
Global \hyperlink{templates_8h_afa8b5c03286c90c5365b874967e81b33}{check\+\_\+xsections} (void)]A careful look at this routine is warranted as get\+\_\+atomic\+\_\+data has changed over the years. 
\item[\label{bug__bug000009}%
\Hypertarget{bug__bug000009}%
Global \hyperlink{templates_8h_abf7699b950ec2ef55a4e0087d82d0ab3}{cooling} (Plasma\+Ptr xxxplasma, double t)]There appear to be redundant calls here to xtotal\+\_\+emission (which include calls to total\+\_\+fb, and to total\+\_\+fb in this routine. This makes it difficult to understand whether something is being counted twice. 
\item[\label{bug__bug000012}%
\Hypertarget{bug__bug000012}%
Global \hyperlink{corona_8c_a547bb990f8958165c0c4382393910d37}{corona\+\_\+rho} (ndom, x)]This and the other routines that describe a coronal model do not handle the case of the vertically extended disk 
\item[\label{bug__bug000013}%
\Hypertarget{bug__bug000013}%
File \hyperlink{cylind__var_8c}{cylind\+\_\+var.c} ]These routines are currently broken. See issue \#159  
\item[\label{bug__bug000015}%
\Hypertarget{bug__bug000015}%
Global \hyperlink{templates_8h_a13144d35b1d57cfa570b6e59d225f474}{cylind\+\_\+volumes} (int ndom, Wind\+Ptr w)]This routine has unresolved X\+XX questions that need to be resolved 
\item[\label{bug__bug000014}%
\Hypertarget{bug__bug000014}%
Global \hyperlink{cylind__var_8c_a3185dc024d7c8bf77fe0be0dc128a272}{cylvar\+\_\+coord\+\_\+fraction} (int ndom, int ichoice, x, ii, frac, int $\ast$nelem)]This routine does not look correct. It does not make any allowance for differences in the z heights of the cells as a function of rho. 
\item[\label{bug__bug000043}%
\Hypertarget{bug__bug000043}%
Global \hyperlink{templates_8h_aa943bd857500a1f2937d8b97d84ff633}{define\+\_\+phot} (Phot\+Ptr p, double f1, double f2, long nphot\+\_\+tot, int ioniz\+\_\+or\+\_\+final, int iwind, int freq\+\_\+sampling)]Is this correct, have subcycles been removed. 
\item[\label{bug__bug000016}%
\Hypertarget{bug__bug000016}%
File \hyperlink{density_8c}{density.c} ]It is not clear why this is a separate file in python. Consider incoprating into another file  
\item[\label{bug__bug000042}%
\Hypertarget{bug__bug000042}%
Global \hyperlink{templates_8h_a8bc3604c533b18b872bb155bfd444bb5}{ds\+\_\+to\+\_\+wind} (Phot\+Ptr pp, int $\ast$ndom\+\_\+current)]1802 -\/ksl -\/ At present this routine for imported models this routine only deals with cylindrical models. Additionally for imported models we skip all of the of the wind\+\_\+cones. This is inefficient, and needs to be corrected for rtheta and spherical models which can easily be handled using wind cones. 
\item[\label{bug__bug000024}%
\Hypertarget{bug__bug000024}%
Global \hyperlink{templates_8h_a280fcbb3dddfaf58a73b2a8d6fd61c76}{extract} (Wind\+Ptr w, Phot\+Ptr p, int itype)]This routine as well as extract\+\_\+one have options for tracking the photon history. The routines are in \hyperlink{diag_8c}{diag.\+c} It is not clear that they have been used in a long time and so it may be worthwhile to remove them. Furthermore, we have established a new mechanism save\+\_\+phot for essentially this same task. This really should be consolidated. 

This is also commented in the text, but there is a rather bizarre separation for where the photon frequency is updated and where the weight is update. The former is done in extract, while the latter is done in extract\+\_\+one. This is not an error precisely but makes the code more confusing than it needs to be. 
\item[\label{bug__bug000023}%
\Hypertarget{bug__bug000023}%
Global \hyperlink{templates_8h_ade7eab896d4188a5b8a4ed6845d33150}{ff} (Wind\+Ptr one, double t\+\_\+e, double freq)]f\+\_\+nu is set to 0 for t\+\_\+e less than 100K. It\textquotesingle{}s not clear that this limit is applied to other functions anymore. Was this missed? Not also that the code having to do with the gaunt factor is duplicated from another routine. Should a gaunt\+\_\+ff routine be created for both? 
\item[\label{bug__bug000053}%
\Hypertarget{bug__bug000053}%
Global \hyperlink{templates_8h_ad350a11e70a4518fb86b842942575e26}{fix\+\_\+concentrations} (Plasma\+Ptr xplasma, int mode)]It is not obvious that this is really what we want. This routine was written when we only were concerned about scattering and does not really take into account emission (for which one needs n\+\_\+e. One possiblity is to calculated ion abundances for H and He properly, in which case we would need the mode variable. 
\item[\label{bug__bug000026}%
\Hypertarget{bug__bug000026}%
Global \hyperlink{get__atomicdata_8c_ae8afa865c507915363062a5abd9c32b7}{get\+\_\+atomic\+\_\+data} (masterfile)]Exactly what is meant by L\+TE here needs clarification.

Needs description -\/ this is a type of data that is no longer used 

This needs a description -\/ it is very old -\/ CK\textquotesingle{}s ground state recombination tables used to compute the zeta term.  
\item[\label{bug__bug000058}%
\Hypertarget{bug__bug000058}%
Global \hyperlink{templates_8h_a897e0ae97c32a481075df9fc45f6e745}{get\+\_\+bl\+\_\+and\+\_\+agn\+\_\+params} (double lstar)]Our long term goal is to make our inputs for radiation sources more generic, but there is still more work to do on this routine. The problems are really to first define what we want. Implementation should be straighforwad if we can figure out what we want. A first step might be to break this into two routines, one for a BL and one for an A\+GN. Note that the A\+GN also is used for a BH binary. This routine was clipped out of python at one pooint so it is not surprising it does not make logical sense to have grouped these itimes together 
\item[\label{bug__bug000018}%
\Hypertarget{bug__bug000018}%
Global \hyperlink{templates_8h_a4ba7a23129d489314ba51c999d18e321}{get\+\_\+extra\+\_\+diagnostics} (void)]This routine should be combined with get\+\_\+standard\+\_\+care\+\_\+factors 
\item[\label{bug__bug000018}%
\Hypertarget{bug__bug000018}%
Global \hyperlink{templates_8h_a4ba7a23129d489314ba51c999d18e321}{get\+\_\+extra\+\_\+diagnostics} (void)]This routine should be combined with get\+\_\+standard\+\_\+care\+\_\+factors 
\item[\label{bug__bug000034}%
\Hypertarget{bug__bug000034}%
Global \hyperlink{templates_8h_abc35742eeeb50cc9cda997c3de567f20}{get\+\_\+knigge\+\_\+wind\+\_\+params} (int ndom)]There may weel be errors associated with a vertically extended disk that need tobe addressed 
\item[\label{bug__bug000017}%
\Hypertarget{bug__bug000017}%
Global \hyperlink{templates_8h_a3c32de99bd46282a95c0de7c8f7e890f}{get\+\_\+standard\+\_\+care\+\_\+factors} (void)]It is not obvious that much recent thought has been given to the choices that are here. The fractional distance that a photon travel is intended to make sure the velocity along the line of sight can be approximated linearly. If a photon travels too far in an azimuthal direction the sense of the velocity can change and this prevensts this 
\item[\label{bug__bug000017}%
\Hypertarget{bug__bug000017}%
Global \hyperlink{templates_8h_a3c32de99bd46282a95c0de7c8f7e890f}{get\+\_\+standard\+\_\+care\+\_\+factors} (void)]It is not obvious that much recent thought has been given to the choices that are here. The fractional distance that a photon travel is intended to make sure the velocity along the line of sight can be approximated linearly. If a photon travels too far in an azimuthal direction the sense of the velocity can change and this prevensts this 
\item[\label{bug__bug000057}%
\Hypertarget{bug__bug000057}%
Global \hyperlink{templates_8h_a5e18aeda5d11190333c93b952b562b04}{get\+\_\+stellar\+\_\+params} (void)]This routine is not completely self consistent, in the sense that for an A\+GN we cannot estimate a luminosity. This reflects the fact that we have not completely setlled on how to homogenize inputs for diffent system types. It\textquotesingle{}s not clear that we need to retun anything. 
\item[\label{bug__bug000057}%
\Hypertarget{bug__bug000057}%
Global \hyperlink{templates_8h_a5e18aeda5d11190333c93b952b562b04}{get\+\_\+stellar\+\_\+params} (void)]This routine is not completely self consistent, in the sense that for an A\+GN we cannot estimate a luminosity. This reflects the fact that we have not completely setlled on how to homogenize inputs for diffent system types. It\textquotesingle{}s not clear that we need to retun anything. 
\item[\label{bug__bug000060}%
\Hypertarget{bug__bug000060}%
Global \hyperlink{templates_8h_a40edef16d314acc2a892cd8b8a5f0eec}{get\+\_\+stellar\+\_\+wind\+\_\+params} (int ndom)]There was an old note here indicating that the ksl was unsure(04j) whether as if the code as written really implements the possibility of a wind that starts at a radius larger than the star. This should be checked 
\item[\label{bug__bug000031}%
\Hypertarget{bug__bug000031}%
File \hyperlink{homologous_8c}{homologous.c} ]X\+X\+XX ksl 1802 -\/ The maximum radius of the wind here seems to be defined externally and it is not clear that this is what one wants in a situation with multiple domains Conisder adding an maximu radius as an imput variable. Note that this may have been fixed  
\item[\label{bug__bug000029}%
\Hypertarget{bug__bug000029}%
Global \hyperlink{get__atomicdata_8c_a8b5d80d6f35b55e30551b5ae6b5cc269}{indexx} (int n, arrin, indx)]Should probably replaces this with the equivalent gsl routine 
\item[\label{bug__bug000019}%
\Hypertarget{bug__bug000019}%
Global \hyperlink{templates_8h_acbb582e8c0def41044e2d4d8aac0c752}{init\+\_\+extra\+\_\+diagnostics} (void)]Ultimately we would like to write the extra diagnositcs to a single file 
\item[\label{bug__bug000019}%
\Hypertarget{bug__bug000019}%
Global \hyperlink{templates_8h_acbb582e8c0def41044e2d4d8aac0c752}{init\+\_\+extra\+\_\+diagnostics} (void)]Ultimately we would like to write the extra diagnositcs to a single file 
\item[\label{bug__bug000054}%
\Hypertarget{bug__bug000054}%
Global \hyperlink{templates_8h_a555f188199cb3346b32e289e64c9d980}{init\+\_\+geo} (void)]Currently init\+\_\+geo is set up for C\+Vs and Stars and not A\+GN. We now read in the system type as the first variable. The intent was to allow one to use the system type to particularize how geo (aqnd other variables) were intialized. But this has yet to be carreid out. 
\item[\label{bug__bug000054}%
\Hypertarget{bug__bug000054}%
Global \hyperlink{templates_8h_a555f188199cb3346b32e289e64c9d980}{init\+\_\+geo} (void)]Currently init\+\_\+geo is set up for C\+Vs and Stars and not A\+GN. We now read in the system type as the first variable. The intent was to allow one to use the system type to particularize how geo (aqnd other variables) were intialized. But this has yet to be carreid out. 
\item[\label{bug__bug000056}%
\Hypertarget{bug__bug000056}%
Global \hyperlink{templates_8h_a3995f2a6ba379027804653e28e4feac1}{init\+\_\+log\+\_\+and\+\_\+windsave} (int restart\+\_\+stat)]This routine was refactored out of \hyperlink{python_8c}{python.\+c} in an attempt to make that routine simpler, but it is not obvious that what done here actually makes sense. For example, although the routine checks if the windsave file exists, and sets restart\+\_\+stat to 0 if it does not. This information is not transmitted back to any other portion of the program. 
\item[\label{bug__bug000035}%
\Hypertarget{bug__bug000035}%
Global \hyperlink{templates_8h_a2ba8dcf484d3688f9687d48dd865dc0d}{levels} (Plasma\+Ptr xplasma, int mode)]Some rethinking of the whole level density approach needs to be done. It\textquotesingle{}s not clear how these functions are being used and what difference if any that they make.  
\item[\label{bug__bug000047}%
\Hypertarget{bug__bug000047}%
Global \hyperlink{templates_8h_a687aa0dcc49f512e562c1e0afddc0263}{mean\+\_\+intensity} (Plasma\+Ptr xplasma, double freq, int mode)]The routine refers to a mode 5, which does not appear to exist, or at least it is not one that is included in \hyperlink{python_8h}{python.\+h} Note also that the logic of this appears overcomplicated, reflecting the evolution of banding, and various ionization modes being added without looking at trying to make this simpler. 
\item[\label{bug__bug000064}%
\Hypertarget{bug__bug000064}%
Global \hyperlink{wind2d_8c_ac98ab21bbcfcfc326f0b16fabd5587c0}{N\+S\+T\+E\+PS} ]This routine has errors. It is only correct if there is a single domain. In particular rho(w,x) gives the correct answer for rho regardless of domains, but nodom is set to 0 for vind(ndom,\&p,v). This should be fixed. 
\item[\label{bug__bug000008}%
\Hypertarget{bug__bug000008}%
Global \hyperlink{templates_8h_aca0d5b8835f560393ae64306e8f24ce3}{one\+\_\+continuum} (int spectype, double t, double g, double freqmin, double freqmax)]The model structure is general in the sense that it was intended for situation with any number of variables. However what is written here is specific to the case of two variables. This is a problem, as we have found in trying to use this in other situations. A more general routine is needed. The first step in doing this is to replace much of the code here with a call to the routine model (spectype, par) in models.\+c 
\item[\label{bug__bug000049}%
\Hypertarget{bug__bug000049}%
Global \hyperlink{templates_8h_a1e3f4af5624039f9a4937939331b5219}{one\+\_\+fb} (Wind\+Ptr one, double f1, double f2)]This routine contains questions from Stuart in May 04 that have never been addressed. Furthemore, the routine has a parameter delta which is used to decide whether one is close enough in temperature to a previously generated D\+CF. This is set to 500, which is probably OK if the temperatures are high, but in appropriate if T is of order 1000 K. 
\item[\label{bug__bug000033}%
\Hypertarget{bug__bug000033}%
Global \hyperlink{templates_8h_a0c0fa3305df1f5c000e0ed0a26d413d8}{one\+\_\+shot} (Plasma\+Ptr xplasma, int mode)]There are some numbered switches included here. This is because these values are not allowed anymore. But this needs to be cleaned up. Deactivated modes should be captured when the parameters are read in, not at this point. An else statement should provide a sufficent check for any unknown mode. 
\item[\label{bug__bug000038}%
\Hypertarget{bug__bug000038}%
File \hyperlink{partition_8c}{partition.c} ]It seems likely that this file and \hyperlink{levels_8c}{levels.\+c} should be combined into a single file. There are very comments (by nsh) about eliminating some of the modes that need investigation as well.  
\item[\label{bug__bug000039}%
\Hypertarget{bug__bug000039}%
Global \hyperlink{templates_8h_a2de3c1f3e0adc389cfda580a1003a688}{partition\+\_\+functions\+\_\+2} (Plasma\+Ptr xplasma, int xnion, double temp, double weight)]According to the historical notes, there is no need for the weight term in the calculation since this is only called when we are making the assumption that we are in L\+TE for the saha equation. 
\item[\label{bug__bug000040}%
\Hypertarget{bug__bug000040}%
Global \hyperlink{templates_8h_a56bcc4eb80e38e626fe58bb5c989eafd}{phot\+\_\+hist} (Phot\+Ptr p, int iswitch)]It is not obvious that this routine is needed as we now have another mechanism to write photons out to a file. In any event, this should be moved to some other place in the code. 
\item[\label{bug__bug000022}%
\Hypertarget{bug__bug000022}%
Global \hyperlink{templates_8h_a5b9038968dd801ac73806fcc16fc1f0c}{photo\+\_\+gen\+\_\+wind} (Phot\+Ptr p, double weight, double freqmin, double freqmax, int photstart, int nphot)]This is another example where the iteration might be made over plasma cells instead of looking for wind cells with positive volume  
\item[\label{bug__bug000046}%
\Hypertarget{bug__bug000046}%
Global \hyperlink{templates_8h_a663f46d36d6962c7993e4c002442c23b}{photon\+\_\+checks} (Phot\+Ptr p, double freqmin, double freqmax, char $\ast$comment)]The routine also a few numbers that summarize some aspects of the the distribution. Some of these have to do with the Ferland definition of IP, which we are discussing deleting 
\item[\label{bug__bug000044}%
\Hypertarget{bug__bug000044}%
Global \hyperlink{templates_8h_addfbba966a4d69bdc71b771e0d7f3cdf}{populate\+\_\+bands} (double f1, double f2, int ioniz\+\_\+or\+\_\+final, int iwind, struct xbands $\ast$band)]f1 and f2 do not appear to be used and should be removed from the call. 
\item[\label{bug__bug000020}%
\Hypertarget{bug__bug000020}%
Global \hyperlink{templates_8h_ada5d5c91372291204ba0255184ce1f87}{read\+\_\+non\+\_\+standard\+\_\+disk\+\_\+profile} (char $\ast$tprofile)]This routine which was written for the Y\+SO study needs to be made less Y\+SO centric. It should also be retested. 
\item[\label{bug__bug000051}%
\Hypertarget{bug__bug000051}%
Global \hyperlink{templates_8h_adc18086c9912f51dae0f13aef98278de}{reposition} (Phot\+Ptr p)]This seems like an error. We have calculated cell specific values of dfudge, but this routine used the global D\+F\+U\+D\+GE to move the photon though a cell wall. 
\item[\label{bug__bug000050}%
\Hypertarget{bug__bug000050}%
File \hyperlink{reposition_8c}{reposition.c} ]This is too short a routine to be in a separate file. It should be incorporated into trans\+\_\+phot or extract which conatin the calling routines.  
\item[\label{bug__bug000001}%
\Hypertarget{bug__bug000001}%
Global \hyperlink{templates_8h_a76f3cece46c92129a1e138a02aac87ca}{reweightwind} (Phot\+Ptr p)]180414 -\/ ksl -\/ It is not obvious that using D\+F\+U\+D\+GE as the way to choose whether to recalculate the cdf for scatter is correct. One would obtain the wrong answer if the photon moved a small distance and hit a different resonance. 

This is a note related to an X\+XX comoment which read as follows\+: Factor of 2 needed because the interval is from -\/1 to 1 X\+XX It\textquotesingle{}s definitely needed to make a uniform distribution work but is this reason really right 
\item[\label{bug__bug000052}%
\Hypertarget{bug__bug000052}%
File \hyperlink{run_8c}{run.c} ]The name of this file is not really acurate. The routines here do drive the major portions of the calculation but they ar still run from \hyperlink{python_8c}{python.\+c}. It might be better to move even more of the running of the code to here. Alternatively, one might make \hyperlink{python_8c}{python.\+c} simpler, so that developers could see the structure better, but moving the input section into it\textquotesingle{}s own file. 
\item[\label{bug__bug000037}%
\Hypertarget{bug__bug000037}%
Global \hyperlink{templates_8h_a4a12de5120dbfd2f67d3ac56645cb4e6}{scattering\+\_\+fraction} (struct lines $\ast$line\+\_\+ptr, Plasma\+Ptr xplasma)]The modes used by scattering fraction are hardwired, which is not our standard approach. Note however there is a lot about the line\+\_\+mode which is quite complicated, not the least of which being that there is a related variable scatter\+\_\+mode 
\item[\label{bug__bug000055}%
\Hypertarget{bug__bug000055}%
Global \hyperlink{templates_8h_ab0e6804f11862cb19cf6ac6976480c3f}{setup\+\_\+dfudge} (void)]This routine does not really belong in this file. It should be moved.  
\item[\label{bug__bug000055}%
\Hypertarget{bug__bug000055}%
Global \hyperlink{templates_8h_ab0e6804f11862cb19cf6ac6976480c3f}{setup\+\_\+dfudge} (void)]This routine does not really belong in this file. It should be moved.  
\item[\label{bug__bug000059}%
\Hypertarget{bug__bug000059}%
Global \hyperlink{templates_8h_a3e05df283bba500d0e7a307679b291d6}{spectrum\+\_\+create} (Phot\+Ptr p, double f1, double f2, int nangle, int select\+\_\+extract)]To create spectra in the live or die option which are selected on the number of scatters individual photons undergo, there are some additional changes required to this routine. These changes should parallel those now in extract under the normal option. 97aug29 
\item[\label{bug__bug000048}%
\Hypertarget{bug__bug000048}%
Global \hyperlink{templates_8h_a9d0d69b33092d06a7174d71c3082d2a6}{total\+\_\+fb} (Wind\+Ptr one, double t, double f1, double f2, int fb\+\_\+choice, int mode)]What is preventing us from calculating a dielectronic emission rate? 
\item[\label{bug__bug000036}%
\Hypertarget{bug__bug000036}%
Global \hyperlink{templates_8h_a0abc40a64ed911b152833da9a675d1b4}{total\+\_\+line\+\_\+emission} (Wind\+Ptr one, double f1, double f2)]It is not exactly clear why two routines total\+\_\+line\+\_\+emission and lum\+\_\+lines is needed, as lum\+\_\+lines appears to be called only from total\+\_\+line\+\_\+emission. ksl -\/ 180509 
\item[\label{bug__bug000041}%
\Hypertarget{bug__bug000041}%
Global \hyperlink{templates_8h_aa1ffd17034054d51819328d892fc57e8}{translate\+\_\+in\+\_\+space} (Phot\+Ptr pp)]There are questions in the comments about why an additonal check is needed as to whther the photon as hit the star. It seems superfluous so someone should check whether this addtional check can be removed. 
\item[\label{bug__bug000062}%
\Hypertarget{bug__bug000062}%
Global \hyperlink{variable__temperature_8c_a72d9769510683e30aeef87c2a255ebe4}{variable\+\_\+temperature} (Plasma\+Ptr xplasma, int mode)]There are various questions associated with this routine about whether what is being done associated with partition funcitons is correct. 
\item[\label{bug__bug000063}%
\Hypertarget{bug__bug000063}%
Global \hyperlink{wind2d_8c_a8fe8eb8bacb888783ada9dd32b523e1a}{wind\+\_\+div\+\_\+err} ]This routine mixes wmain (which is passed externally) and w which is passed by calling it. While this does not generate an error it should be cleaned up, proably to use only wmain 
\item[\label{bug__bug000021}%
\Hypertarget{bug__bug000021}%
Global \hyperlink{templates_8h_ae1d27b840ad2ff21462921fc2aab975c}{wind\+\_\+luminosity} (double f1, double f2)]The do loop might be simpler if made over the plasma cells instead of the wind, but one should be careful of the dummy cell 
\item[\label{bug__bug000065}%
\Hypertarget{bug__bug000065}%
Global \hyperlink{windsave_8c_a1bf368d85f72b220cc4cba091a5e0702}{wind\+\_\+read} (filename)]This routine calls wind\+\_\+complete, but it is not entirely clear why, as the values calculated there are already in the domain structure, so this seems redundant. However there is an issue \#41 which has to do with reading in windcones that affects this.  
\item[\label{bug__bug000061}%
\Hypertarget{bug__bug000061}%
Global \hyperlink{util_8c_ad2938cb78e3414274491d2d778843db3}{wind\+\_\+x\+\_\+to\+\_\+n} (double x\mbox{[}\mbox{]}, int $\ast$n)]There are no checks for whether x is within the wind. This seems very dangerous. 
\item[\label{bug__bug000005}%
\Hypertarget{bug__bug000005}%
Global \hyperlink{bilinear_8c_af582498786ae60301965c85d405d7cde}{xquadratic} (double a, double b, double c, r)]xquadratic looks line for line identical to another routine quadratic which can be found in \hyperlink{phot__util_8c}{phot\+\_\+util.\+c}. Both versions of the code seem to be called. One of them should be removed. 
\item[\label{bug__bug000010}%
\Hypertarget{bug__bug000010}%
Global \hyperlink{templates_8h_af3162ce29644bbfd1d6ef9060d972643}{xtotal\+\_\+emission} (Wind\+Ptr one, double f1, double f2)]The call to this routine was changed when Plasma\+Ptrs were introduced, but it appears that the various routines that were called were not changed. This needs to be fixed for consistency
\end{DoxyRefList}